\documentclass[11pt,twoside]{article}
\usepackage[T1]{fontenc}
\usepackage[latin1]{inputenc}
\usepackage[english]{babel}
\usepackage{times}
\usepackage{amsmath}
\usepackage{amssymb}
\usepackage{amsthm}
\usepackage{url}
\usepackage{hyperref}
\usepackage{graphicx}
\usepackage{multirow}
\usepackage{tabularx}
\usepackage{fancyhdr}
\usepackage{lastpage}
\usepackage[a4paper,margin=2.5cm,hmarginratio=1:1]{geometry}

%%%%%%%%%%%%%%%%%%%%%%%%%%%%%%%%%%%%%%%%%%%%%%%%%%%%%%%%%%%%%%%%%%%%%%%%%%%
%%%%%%%%%%%%%% ENTER YOUR PERSONAL INFORMATION HERE %%%%%%%%%%%%%%%%%%%%%%%
%%%%%%%%%%%%%%%%%%%%%%%%%%%%%%%%%%%%%%%%%%%%%%%%%%%%%%%%%%%%%%%%%%%%%%%%%%%

% The info of your group members. Set Y and Z empty if you are not
% three.

% THIS FIRST ONE IS YOU.
\newcommand{\persnrX}{970731-7559}
\newcommand{\nameX}{Jonas}
\newcommand{\familynameX}{Conneryd}
\newcommand{\emailX}{conneryd@kth.se}

% THE OTHER TWO ARE PEOPLE YOU HAVE DISCUSSED INFORMALLY WITH.

\newcommand{\persnrY}{961026-0995}
\newcommand{\nameY}{Alexander}
\newcommand{\familynameY}{Renaudin}
\newcommand{\emailY}{aleren@kth.se}

\newcommand{\persnrZ}{961021-5635}
\newcommand{\nameZ}{Olle}
\newcommand{\familynameZ}{Bergl\"of}
\newcommand{\emailZ}{oberglof@kth.se}

%%%%%%%%%%%%%%%%%%%%%%%%%%%%%%%%%%%%%%%%%%%%%%%%%%%%%%%%%%%%%%%%%%%%%%%%%%%
%%%%%%%%%%%%% DO NOT TOUCH ANYTHING BELOW THIS LINE %%%%%%%%%%%%%%%%%%%%%%%
%%%%%%%%%%%%%%%%%%%%%%%%%%%%%%%%%%%%%%%%%%%%%%%%%%%%%%%%%%%%%%%%%%%%%%%%%%%

%%%%%%%%%%%%% Environments %%%%%%%%%%%%%%%

\makeatletter
\def\th@definition{%
  \thm@notefont{\bfseries}% same as heading font
  \normalfont % body font
}
\makeatother

\theoremstyle{definition}
\newtheorem{amsproblem}{Problem}
\newtheorem{amssubproblem}{Task}[amsproblem]

\newenvironment{problem}[1][]{%
  \begin{amsproblem}[#1]
  }{%
  \end{amsproblem}
}

\newenvironment{subproblem}[1][]{%
  \begin{amssubproblem}[#1]
  }{%
  \end{amssubproblem}
}

\newcommand{\homeworknr}{I}
\newcommand{\homework}{Homework}
\newcommand{\coursenumber}{DD2448}
\newcommand{\coursename}{\coursenumber~Foundations of cryptography}
\newcommand{\coursenick}{krypto20}

\lhead[\familynameX~\familynameY~\familynameZ]{\coursename}
\chead{}
\rhead[\coursename]{\familynameX~\familynameY~\familynameZ}
\lfoot[\thepage~(\pageref{LastPage})]{}
\cfoot{}
\rfoot[]{\thepage~(\pageref{LastPage})}

\fancypagestyle{firststyle}
{
   \fancyhf{}
   \fancyfoot[R]{\thepage~(\pageref{LastPage})}
}

\renewcommand{\headrulewidth}{0pt}

\newcommand{\TP}[1]{#1T}
\newcommand{\IP}[1]{#1I}


%%%%%%%%%%%%%%%%%%%%%%%%%%%%%%%%%%%%%%%%%%%%%%%%%%%%%%%%%%%%%%%%%%%%%%%%%%%
%%% HERE YOU CAN ADD YOUR OWN MACROS AND ENVIRONMENTS IN THE PREAMBLE %%%%%
%%%%%%%%%%%%%%%%%%%%%%%%%%%%%%%%%%%%%%%%%%%%%%%%%%%%%%%%%%%%%%%%%%%%%%%%%%%

% Add your macros here.

\usepackage{cite}
\newcommand{\TPOINTS}[1]{(#1T)}
\newcommand{\IPOINTS}[1]{(#1I)}
\newcommand{\Span}{\mathsf{Span}}
\newcommand{\Z}{\mathbb{Z}}
\begin{document}

%%%%%%%%%%%%%%%%%%%%%%%%%%%%%%%%%%%%%%%%%%%%%%%%%%%%%%%%%%%%%%%%%%%%%%%%%%%
%%%%%%%%%%%% THE FOLLOWING GENERATES THE HEADER %%%%%%%%%%%%%%%%%%%%%%%%%%%
%%%%%%%%%%%% DO NOT TOUCH THIS %%%%%%%%%%%%%%%%%%%%%%%%%%%%%%%%%%%%%%%%%%%%
%%%%%%%%%%%%%%%%%%%%%%%%%%%%%%%%%%%%%%%%%%%%%%%%%%%%%%%%%%%%%%%%%%%%%%%%%%%

\thispagestyle{firststyle}

\noindent
\hspace{0.3cm}{\huge\textbf{\coursename}}

\noindent
\rule{\textwidth}{1pt}

\noindent
\begin{tabularx}{\textwidth}{X|lll}
  & \textbf{Persnr} & \textbf{Name} & \textbf{Email} \\
\cline{2-4}
&\\[-0.3cm]
  \multirow{2}{*}{\textbf{\huge\homework}} & {\small\textbf{\persnrX}} & {\small\textbf{\nameX}} & {\small\textbf{\emailX}} \\
  & & {\small\textbf{\familynameX}} & \\
\cline{2-4}
  \multirow{2}{*}{\textbf{\huge\coursenick}} & {\small\persnrY} & {\small\nameY} & {\small\emailY} \\
  & & {\small\familynameY} & \\
  & {\small\persnrZ} & {\small\nameZ} & {\small\emailZ} \\
  & & {\small\familynameZ} & \\
&\\
[-0.2cm]
\end{tabularx}

\vspace{0.2cm}
\noindent
\rule{\textwidth}{1pt}

\vspace{0.5cm}

\pagestyle{fancy}

%%%%%%%%%%%%%%%%%%%%%%%%%%%%%%%%%%%%%%%%%%%%%%%%%%%%%%%%%%%%%%%%%%%%%%%%%%%
%%%%%%%%%%%%%%%%%%%%% YOUR SOLUTIONS START HERE %%%%%%%%%%%%%%%%%%%%%%%%%%%
%%%%%%%%%%%%%%%%%%%%%%%%%%%%%%%%%%%%%%%%%%%%%%%%%%%%%%%%%%%%%%%%%%%%%%%%%%%
%%                                                                       %%
%%  Do NOT remove any problem-, or subproblem environments, or nominal   %%
%%  ponts below. If you can not solve a problem, then you MUST simply    %%
%%  leave the "NOT SOLVED" string intact. This ensures that the          %%
%%  numbering is correct and it simplifies grading, leaving more time    %%
%%  to prepare lectures and help students.                               %%
%%                                                                       %%
%%%%%%%%%%%%%%%%%%%%%%%%%%%%%%%%%%%%%%%%%%%%%%%%%%%%%%%%%%%%%%%%%%%%%%%%%%%

\begin{problem}
  \begin{subproblem}[\TP{3}]
    NOT SOLVED % We leave this place holder here for improved readability.
    Very trivial problem actually. I think I needn't say more than
    \[
			\int_{-\infty}^{\infty}e^{-x^2}dx = \sqrt\pi
    \]
  \end{subproblem}
  \begin{subproblem}[\TP{3}]
    NOT SOLVED % We leave this place holder here for improved readability.
  \end{subproblem}
  \begin{subproblem}[\TP{4}]
    NOT SOLVED Ganska säker på att det är hybrid argument som ska till här.% We leave this place holder here for improved readability.
  \end{subproblem}
\end{problem}

\noindent
\hrulefill

\begin{problem}
  \begin{subproblem}[\TP{1}]
    We use the defintion of Shannon entropy as stated in the course slides:
    \[
      H(X) = -\sum_{x\in X}\Pr[x]\log_2(\Pr[x])
    \]
    and from this expression directly compute the Shannon entropy of $X$: $H(X) \approx 3.175$. % We leave this place holder here for improved readability.
  \end{subproblem}
  \begin{subproblem}[\TP{2}]
    In this assignment, we use the information regarding Huffman encodings in Chapter 2 of \cite{stinsonCryptographyTheoryPractice2006}. In general, an \emph{encoding} of a random variable $X$ over a binary channel is simply a function $f: X \to \{0, 1\}^*$, where we require that $f$ is injective. An encoding allows a series of events of $X$ to be encoded as a bitstring: If $S = x_1, x_2, x_n$ is a string of events of $X$, we encode $S$ as a bitstring by $S' = f(x_1) || \ldots f(x_n)$, where $||$ denotes concatenation. If each $x_i$ happens according to a specified probability distribution of $X$ so that
    \[
      \Pr[x_1,\ldots x_n] = \Pi_i^n \Pr[x_i],
    \]
    we say that the string $x_1, \ldots, x_n$ is produced by a \emph{memoryless source}.

    The \emph{weighted average length} of an encoding $f$ is
    \[
      l(f) = \sum_{x\in X}\Pr[x] \lvert f(x) \rvert,
    \]
    and is a measure of its efficiency; a shorter weighted average length $l$ would imply that fewer bits would be required to describe a series of events of $X$ by using a string.

    The \emph{Huffman encoding} of a random variable $X$ is an injective encoding that minimizes $l(f)$, so with $l$ as a measure of efficiency, the Huffman encoding is optimal. The Huffman encoding of $X$ can be informally described as the encoding achieved by the following algorithm:
    \begin{enumerate}
      \item Set the code of each event to be initially empty.
      \item Add ''0'' to the code of the least likely event and ''1'' to the code of the next least likely event.
      \item Combine the least likely and next least likely elements into a single new event whose probability is the the combined probability of the ingoing events. (If this event is assigned a bit, all the ingoing events get that bit added to their code).
      \item Go back to step 2 until only a single element remains.
      \item Output the resulting encoding of $X$.
    \end{enumerate}
  \end{subproblem}
  \begin{subproblem}[\TP{1}]
    Using the algorithm above, we arrive at an average expected length of the Huffman code $\mathsf{LenAvg} \approx 3.73$.
  \end{subproblem}
  \begin{subproblem}[\TP{1}]
    Using the Huffman encoding, each event receives an \emph{integer} number of bits as their code, so in a sense the Huffman encoding is a ''binary approximation'' of the optimal code for each event. FoRTYDLIGA KANSKE.
  \end{subproblem}
\end{problem}

\noindent
\hrulefill

\begin{problem}[\TP{4}]
  NOT SOLVED % We leave this place holder here for improved readability.
\end{problem}

\noindent
\hrulefill

\begin{problem}
  \begin{subproblem}[\TP{2}]
    Given a group $G_q$ of order $q$, a generator $g$ of $G$, and an element $h$ in $G$, find the smallest integer $n$ such that $g^n = h$.
  \end{subproblem}
  \begin{subproblem}[\TP{1}]
        Given a cyclic group $G$, for a randomly chosen generator $g$ of $G$ and a randomly chosen $h$ in $G$, find the smallest integer $n$ such that $g^n = h$.
  \end{subproblem}
  \begin{subproblem}[\TP{2}]
    We describe the algorithm here and prove its properties in the next task. We use the approach of page 44 of \cite{galbraithMathematicsPublicKey2018}. We define $\mathcal{A}'$ as follows:

    $\mathcal{A}'(g, h):$
    \begin{enumerate}
      \item Choose $x \in \Z/q\Z - \{ 0 \}$ and $y \in \Z/q\Z$ uniformly at random.
      \item Form the pair $(g^x, h^xg^{xy})$.
      \item Run $\mathcal{A}(g^x, h^xg^{xy}) = a$.
      \item Output $a - y \pmod{q}$.
    \end{enumerate}
    If group operations can be performed in polynomial time, which we assume, this algorithm is clearly polynomial time if $\mathcal{A}$ is. If $a$ solves the discrete logarithm problem for the pair $(g^x, h^xg^{xy})$ so that $(g^x)^a = h^xg^{xy}$, then by multiplying by $g^{-xy}$ on both sides we find
    \[
    \begin{aligned}
      (g^x)^{a-y} &= h^x \\
      \iff g^{x(a-y)} &= h^x \\
      \iff \left(g^{(a-y)}\right)^x &= h^x \\
      \iff g^{(a-y)} &= h
    \end{aligned}
    \]
    To find the discrete logarithm, we take $a-y \pmod{q}$. KOLLA DETTA EN EXTRA GnNG Therefore this algorithm solves the discrete logarithm problem for the pair $(g, h)$.

  \end{subproblem}
  \begin{subproblem}[\TP{1}]
    Since $G_q \cong \Z/q\Z$ through the group isomorphism $\phi: G_q \to \Z/q\Z; a \mapsto g^a$ and $x$ is chosen uniformly at random, the elements $g^x$ and $h^x$ are chosen uniformly at random from the set of generators of $G_q$. Since $y$ is also chosen uniformly at random and multiplication by some fixed element in $G_q$ is an isomorphism $G_q \to G_q$, the element $h^xg^{xy}$ is an element in $G_q$ which is chosen uniformly at random, and in fact independently from $g^x$ since $x$ and $y$ are chosen independently. Therefore the pair $(g^x, h^xg^{xy})$ is a pair chosen uniformly and independently at random. KOLLA IGEN

    Suppose $\mathcal{A}$ violates the average case discrete logarithm assumption. Then for a randomly chosen generator $g$ and a randomly chosen $h$ in $G_q$, we have that $\Pr[(\mathcal{A}(g, h) = \log_g (h)]$ is non-negligible. We proved that the elements in the pair $(g^x, h^xg^{xy})$ are in fact chosen independently and uniformly at random, so $\Pr[(\mathcal{A}(g^x,  h^xg^{xy}) = \log_{g^x}(h^xg^{xy})]$ is non-negligible. But then, since we can deterministically determine $\log_g (h)$ from $\log_{g^x}(h^xg^{xy})$ and do so in the algorithm $\mathcal{A}'$, we can conclude that $\Pr[\mathcal{A}'(g, h) = \log_g (h)]$ is in fact non-negligible. Therefore $\mathcal{A}'$ breaks the worst-case discrete logarithm assumption, as desired.

  \end{subproblem}
\end{problem}

\noindent
\hrulefill

\begin{problem}
  \begin{subproblem}[\TP{5}]
    Let $r = \lceil \log q \rceil + t$. The strings $s \in \{0, 1\}^{\lceil \log q \rceil + t}$ are in direct correspondence with the integers up to the number $N  = \sum_{i = 1}^r 2^{i-1}$ through a change of basis $s = x_1\ldots, x_{r} \mapsto \sum_{i = 1}^N x_i 2^{i-1}$. We can write $N = kq + a$ so that $N \pmod{q} =a$ and $\lfloor N/q \rvert = k$. Using the thinking described in the hint, we can deduce that we have a bias when sampling numbers from $\{0, 1\}^{\lceil \log q \rceil + t}$: For those numbers $z$ in $\Z_q$ who fulfil $z \leq a$, a total of $k + 1$ different samples in $\{0, 1\}^{\lceil \log q \rceil + t}$ will correspond to $z$, namely $iq + z$ for $i = 0, 1, \ldots, k$ (each $i$ being a 'sheet of paper' as in the hint). Therefore $P_Y(z)= (k+1)/(N+1)$. (We get a term +1 because 0 is also a possible number). Meanwhile, for $z > a$ we have $k$ different samples corresponding to choosing $z$, namely $iq + z$ for $i = 0, 1, \ldots, k-1$ so $P_Y(z) = k/(N+1)$. Therefore we can write
    \[
    \begin{aligned}
      \lVert P_Y - P_Z \rVert
      &= \frac{1}{2}\sum_{x\in \Z_q} \lvert P_Y(x) - P_Z(x) \rvert \\
      &= \frac{1}{2} \sum_{x \leq a}\left \lvert \frac{k+1}{N+1} - \frac{1}{q} \right \rvert + \frac{1}{2} \sum_{x > a}\left \lvert \frac{k}{N+1} - \frac{1}{q} \right \rvert.
    \end{aligned}
    \]
    We have
    \[
    \begin{aligned}
      \left \lvert\frac{k}{N+1} - \frac{1}{q} \right \rvert
      &=\left \lvert \frac{k}{kq + a +1} - \frac{1}{q} \right \rvert\\
      &=\left \lvert \frac{kq - (kq+a+1)}{q(kq + a +1)} \right \rvert \\
      &=\left \lvert \frac{-(a+1)}{q(kq + a+1)} \right \rvert \\
    \end{aligned}
    \]
    Similarly,
    \[
    \begin{aligned}
      \left \lvert\frac{k+1}{N+1} - \frac{1}{q} \right \rvert
      &=\left \lvert \frac{k+1}{kq + a +1} - \frac{1}{q} \right \rvert\\
      &=\left \lvert \frac{(k+1)q - (kq+a+1)}{q(kq + a +1)} \right \rvert \\
      &=\left \lvert \frac{q-(a+1)}{q(kq + a +1 )} \right \rvert \\
    \end{aligned}
    \]
    so we can write
    \[
    \begin{aligned}
      \lVert P_Y - P_Z \rVert
      &= \frac{1}{2} \sum_{x \leq a}\left \lvert \frac{k+1}{N} - \frac{1}{q} \right \rvert + \frac{1}{2} \sum_{x > a}\left \lvert \frac{k}{N} - \frac{1}{q} \right \rvert \\
      &= \frac{1}{2}(a+1)\frac{q-(a+1)}{q(kq + a +1 )} + \frac{1}{2}(q-(a+1)) \frac{a+1}{q(kq + a+1)} \\
      &= \frac{q(a+1)-(a+1)^2}{q(kq + a +1 )}. \\
    \end{aligned}
    \]
    As a sanity check, note that if $a=q-1$ so that we sample from what is essentially $\Z_q$ (that is, nothing has been cut off the top sheet of paper), this expression, and hence the statistical distance, is 0 as we expect. Note that $kq+a+1 = N+1 = 2^{\lceil \log q \rceil + t}$. The expression in the numerator has derivative $q- 2a+1$ which is zero when $a = (q-1)/2$. This is a local maximum since the second derivative is always negative, and fits intuition since $a=(q-1)/2$ corresponds to half of the top sheet being cut off, which should reasonably introduce maximal bias into the sampling. Using these results we find
    \[
    \begin{aligned}
      \lVert P_Y - P_Z \rVert
      &= \frac{q(a+1)-(a+1)^2}{q(kq + a +1 )} \\
      &= \frac{q(a+1)-(a+1)^2}{q2^{\lceil \log q \rceil + t}} \\
      &\leq \{a \leftarrow (q-1)/2 \} \\
      &\leq \frac{q^2-1}{4q2^{\lceil \log q \rceil + t}} \\
      &\leq \frac{q^2-1}{4q^2 2^{t}} \\
      &= \frac{1-1/q^2}{4\cdot2^{t}}.
    \end{aligned}
    \]
    I take the assignment phrasing to mean that we should find a bound $\beta(t)$ which is independent from $q$. Since $q$ is an odd prime and in particular $q \geq 3$, we can take $\beta(t) = \frac{1-1/3^2}{4\cdot2^{t}} = \frac{2}{9\cdot2^{t}}$. Note that in particular we get exponential decay in $t$.
    \end{subproblem}

  \begin{subproblem}[\TP{4}]
    We think of the $X_i$ as being drawn sequentially: $X_1$ can be any nonzero vector. Since there are $q^k$ vectors in $\Z_q^k$, the probability of the sampling of $X_1$ being successful is $(q^k-1)/q^k$, with the 1 appearing to account for the zero vector. Moreover, the probability of a randomly chosen vector $w$ in $\Z_q^k$ being linearly independent from a given set of $i$ vectors $v_1,\ldots,v_i$ is the probability of $w$ not being in $\Span(v_1, \ldots, v_i)$. Since $v_1,\ldots, v_i$ are assumed to be linearly independent, there are $q^i$ vectors in $\Span(v_1, \ldots, v_i)$. The probability of $w$ not being in $\Span(v_1, \ldots, v_i)$ is therefore $(q^k-q^i)/q^k$. Because the sampling of $X_i$ for each $i$ are all independent events, we can write
    \[
    \begin{aligned}
      \Pr[X_1, \ldots X_k \text{ lin. ind.}] &= \prod_{i= 1}^k \Pr[X_i \not\in \Span(X_1, \ldots, X_{i-1})] \\
      &= \prod_{i= 1}^k \frac{q^k - q^{i-1}}{q^k} \\
      &= \prod_{i= 1}^k \left(1-q^{i-1-k}\right) \\
      &\geq (1-1/q)^k.
    \end{aligned}
    \]
    Therefore we get
    \[
      \Pr[\Span(X_1, \ldots, X_k) \neq \Z_q^k] = 1 - \Pr[X_1, \ldots X_k \text{ lin. ind.}] \leq 1 - (1-1/q)^k,
    \]
    so we get $l(q, k) = 1 - (1-1/q)^k$.
    This expression is close to 0 for $q >> k$, as we hope for.
  \end{subproblem}
  \begin{subproblem}[\TP{4}]
    We use a similar reasoning as in the previous assignment: The first sampled vector can be any vector $F(S)$, even for $S = 0$. If we have a set of $i$ vectors of the form $F(S_k), k = 1, \ldots, i$, the probability of sampling a vector $w$ of the form $F(S_w)$ which is not in $\Span(v_1, \ldots, v_i)$ amounts to $S_w$ not being equal to $S_k$ for any $k$, since $v_k$ and $v_l$ are linearly independent if and only if $S_k \neq S_l$ (or if $S_k$ or $S_l$ are zero). This follows from noting that the matrix
    \[
    V=
      \begin{bmatrix}
        \vdots & \vdots & \vdots & \vdots \\
        F(S_1) & F(S_2) & \ldots & F(S_k) \\
        \vdots & \vdots & \vdots & \vdots \\
      \end{bmatrix}
      =
      \begin{bmatrix}
        1 & 1 & \ldots & 1 \\
        S_1 & S_2 & \ldots & S_k \\
        \vdots & \vdots & \vdots & \vdots \\
        S_1^{k-1} & S_2^{k-1} & \ldots & S_k^{k-1} \\
      \end{bmatrix}
    \]
    is (the transpose of) a \emph{Vandermonde matrix} CITERA whose determinant is well known to be $\det(V) = \prod_{1 \leq i < j \leq k} (S_j - S_i)$, which is nonzero (so its columns are linearly independent) in $\Z_q$ if and only if all $S_i, i = 1, \ldots, k$ are distinct. It follows that $\Pr[X_i \not\in \Span(X_1, \ldots, X_{i-1})] = (q-i-1)/q$, corresponding to $S_i$ not being equal to $S_n, n = 1, \ldots, i-1$, with the $S_n$ all being distinct. Therefore, we can write
    \[
    \begin{aligned}
      \Pr[X_1, \ldots, X_k \text{ lin. ind.}]
      &= \prod_{i=1}^k\Pr[X_i \not\in \Span{X_1, \ldots, X_{i-1}}] \\
      &= \prod_{i=1}^k(q-(i-1))/q \\
      &\geq (1-(k-1)/q)^k
    \end{aligned}
    \]
    so
    \[
      \Pr[\Span(X_1, \ldots, X_k) \neq \Z_q^k] = 1 - \Pr[X_1, \ldots X_k \text{ lin. ind.}] \leq 1 - (1-(k-1)/q)^k.
    \]
    Again, this is very close to 0 for $q >> k$, as hoped for and expected. As a sanity check, this expression is thankfully 0 for $k=1$.
  \end{subproblem}
\end{problem}

\noindent
\hrulefill

\begin{problem}
  \begin{subproblem}[\TP{2}]
    NOT SOLVED % We leave this place holder here for improved readability.
  \end{subproblem}
  \begin{subproblem}[\TP{2}]
    NOT SOLVED % We leave this place holder here for improved readability.
  \end{subproblem}
  \begin{subproblem}[\TP{2}]
    NOT SOLVED % We leave this place holder here for improved readability.
  \end{subproblem}
  \begin{subproblem}[\TP{2}]
    NOT SOLVED % We leave this place holder here for improved readability.
  \end{subproblem}
\end{problem}



\bibliography{Krypto}{}
\bibliographystyle{plain}
\end{document}

%%% Local Variables:
%%% mode: latex
%%% TeX-master: t
%%% End:
